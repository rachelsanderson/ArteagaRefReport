\documentclass[a4paper, 11pt]{article}
\usepackage{comment} % enables the use of multi-line comments (\ifx \fi)
\usepackage{lipsum} %This package just generates Lorem Ipsum filler text.
\usepackage{fullpage} % changes the margin
\usepackage{indentfirst}
\usepackage[utf8]{inputenc}
\usepackage{booktabs}
\usepackage{chngpage}
\setlength{\parindent}{2em}
\renewcommand{\baselinestretch}{1.1} %make line spacing bigger
%custom commands
%\usepackage{pgfplots}
\usepackage[margin=1in]{geometry}

\begin{document}
%Header-Make sure you update this information!!!!

\title{Referee Report for the Quarterly Journal of Economics \\
\Large ``The effect of human capital on earnings: Evidence from a reform at Columbia's top university"}
\author{Rachel Anderson}
\maketitle

\section*{Overview}

The paper aims to test whether the returns to college education result from human capital accumulation or the fact that attending college signals higher ability to employers.  The author does this by studying the effects of a curriculum reform at a top Colombian university that reduced the number of required courses for its economics and business programs by 20\% and 14\% respectively; but did not change the quality or amount of incoming and graduating students.  She estimates that after the reform, graduates' wages fell by approximately 16\% in economics and 13\% percent in business, which she claims to be a result of less human capital accumulation.  Exploring the mechanisms through which wages declined, the author suggests that students performed worse in the recruitment process, which led them to be placed in lower-quality firms.  Specifically, for the economist recruitment process at the Central Bank of Colombia, she finds that the reform reduced the probability of being hired by 17 percentage points. 

\section*{Contribution}

This is not the first paper to attempt to estimate the effect of human capital accumulation on wages, separate from that of signaling; however, it is one of the first to do so convincingly, and in a college setting.  Previous papers on the topic focusing on primary and secondary school settings have achieved mixed results, and generally support a signaling framework (\cite{hu16}, \cite{bedard01}, \cite{lang86}).  A related strand of literature seeks to measure directly whether there is signaling value to academic degrees (Clark and Martorell; Murnane and Willett).  Although it cannot resolve these inconsistencies in the literature, this paper contributes a convincing piece of evidence in favor of human capital theory, and provides important information about the tools employers use to learn about workers' expected productivity.  These findings also contribute to a third strand of literature that estimates the returns to different types of post-secondary education by separating the effect of the institution versus the field of study (Dale and Krueger); this paper suggests that heterogeniety in returns may be due to different sets of skills in each degree, not due to differences in selectivity.


\section*{A threat to identification}

The key identification assumption is that the curriculum reform did not alter the selection of entering or graduating students at Los Andes.  If this is true, then a pure ``signaling" model predicts that the reform should not affect the wages of new graduates because there is no change in the unobserved ability of students.  By contrast, the human capital model predicts that wages should decline after the reform, due to a decline in the value added from the economics and business programs at Los Andes. 

While the author does provide evidence that the composition of entering and graduating students has not changed on dimensions such as high school exit exam performance, she overlooks an institutional detail poised to threaten her results: the introduction of national college exit exams in Colombia.

In 2004, the agency that administers the college admission exam in Colombia introduced field-specific exit exams to be administered in every college offering a related program.  The stated intent of this change was to improve accountability in the college market, so that school-level aggregate scores could be made available and used by news outlets as part of college rankings.  Different degree programs offered exams beginning in different years, however both business and economics were among the first to offer exams in 2004. 

The curriculum change at Universidad de Los Andes studied in the paper under review occurred in 2006, so that the pre- and post-reform groups are likely to have been directly affected by the introduction of national exit exams.  For both groups of students, exit exam performance may serve as another signal of ability, since many graduates list scores on their CVs and online profiles.  Additionally, exit exams may affect faculty recommendations or students' job search behavior after learning their position in the national distribution of exam takers.  

Support for this hypothesis comes from MacLeod et al (2017), who exploit heterogeneity in exit exam rollout across programs to conclude that the return to college reputation, as measured by the mean admission score of a school's graduates, declined in the presence of a new measure of individual skill.  The authors also observed decreasing returns to college admission exam scores, as a result of increasing the returns to ability, as measured by college exit exam performance. 

Table \ref{tab:macleod} reproduces specific results for economics and business programs from MacLeod et al (2017).  Compared to cohorts enrolling in these programs between 2003-2004, the return to college reputation on log average daily wages for students enrolling in 2005-09 declined somewhere between 5 and 10 percent.  The returns to ability, here defined as individual college admission scores, also declined slightly over this period, due to the presence of a new signal.   

\begin{table}[h!]
 \begin{adjustwidth}{-.75in}{-.75in} 
 \begin{center}
 \caption{Returns to reputation and ability by program and cohort on log average daily earnings (MacLeod et al. 2017)}
 \vspace{8pt}
\begin{tabular}{cccccccccc}
\toprule
             &                    &        &          & \multicolumn{3}{c}{Return to reputation}&\multicolumn{3}{c}{Return to ability}       \\
             \cmidrule(lr){5-7} \cmidrule(lr){8-10}
Program area & Program            & N      & Colleges & 2003-04              & 2005-09 & Diff. & 2003-04           & 2005-09 & Diff. \\
\midrule
Business     & Administraci\'on     & 85,325 & 46       & 0.18                 & 0.14    & -0.05 & 0.04              & 0.03    & -0.01 \\
Business     & Contadur\'ia p\'ublica & 49,714 & 36       & 0.18                 & 0.09    & -0.09 & 0.03              & 0.03    & 0.00  \\
Business     & Econom\'ia           & 25,879 & 21       & 0.21                 & 0.11    & -0.10 & 0.05              & 0.03    & -0.02 \\
\bottomrule
\multicolumn{8}{l}{\small{*Cohort refers to graduating cohort. This is replication of TABLE X FROM APPENDIX}}
\end{tabular}
\label{tab:macleod}
\end{center}
\end{adjustwidth}
\end{table}

\section*{Potential implications on results}

The author's baseline database contains all students who started college between 2002 and 2007 and graduated after 2004 from Universidad de Los Andes and other top-10 schools in Colombia.  Yet the results from Table  \ref{tab:macleod} suggest that members of both the baseline and treatment groups would have been affected by the introduction of exit exams, and that these effects are heterogeneous across cohorts.  Without controlling for the proportion of students taking the exit exams or the average score among those  graduates at each school, it is impossible to separate the effect of a curriculum change at Los Andes from trends in the broader college market.  


Using the work of MacLeod et al (2017) as a new lens for analysis, I devote the remainder of this section to listing some of the potential implications of this threat to identification on each of the author's results and robustness checks.

\subsection*{Main Result}

Direct comparisons with MacLeod et al (2017) are made difficult by the fact that the author uses ``cohort" to refer to the semester and year the students started school; thus it is challenging, if not impossible, to infer which ``cohorts" would have been exposed to the exit exams.  College exit exams did not become mandatory for graduation until 2009, however test-taking increased rapidly from 2004-2009 -- for example, from 10 percent of students graduating in 2004 to 55 percent of students graduating in 2005.  If students took around 4-4.5 years to graduate, the lines that indicate the time of the curriculum reform in Figure 4 also coincide with the first cohort for whom the exit exams would have been mandatory.  

Thus, without controlling for exit exam participation and performance, the main result of the paper that wages fell by approximately 16\% in economics and 13\% in business may be explained by any of the following competing hypotheses: 
\begin{enumerate}
\item Prior to 2009, Los Andes students were less likely to take exit exams than students at other universities.  Cohorts graduating prior to 2009 continued to benefit from high returns to the Los Andes reputation.  After the exams became mandatory, Los Andes students began performing worse on the job market relative to students at other schools, due to the presence of a new signal about student ability.  Given the context of the paper, Los Andes students may have signaled lower ability because of any of the following:
\begin{enumerate}
\item The curriculum reform in 2006, which left Los Andes students less prepared for the exit exams relative to students at other schools.  This is supported by the fact that the author writes Los Andes was the only school to implement a curriculum reform at this time.
\item Other schools may have changed curricula in less obvious ways, such as adding test preparation sessions, or motivating individuals to work harder in preparation for the exams.  \item Without reporting the correlation between students' scores on the two exams, we cannot assume that students performing well on the college admissions exam also perform well on the exit exam.  Furthermore \textit{regression to the mean} suggests that if a variable is extreme on its first measurement (i.e. students who receive top scores on the college admissions exam), will tend to be closer to the average on the second measurement.  

\end{enumerate}

\item Alternatively, Los Andes students may have been equally or more likely to take exit exams prior to 2009, but their relative percentile rank declined after the exams became mandatory due to the market becoming more competitive.  
\begin{enumerate}
\item In this case, the introduction of exit exams may have changed students' strategy for finding a job.   Students, after observing where they score relative to other students, may change the types of jobs where they apply to, based on where they believe they are competitive.  
\item Prior to the exit exams, students at Los Andes may have benefited from alumni/faculty connections at top firms.  The introduction of exit exams made students believe the market was more competitive.  Similarly, faculty may have changed how they choose to recommend students for certain jobs, according to how students perform on the exams. 
\item The wage decline observed in Figure 4 could reflect employer's learning about the relationship between test scores and worker performance. 
\end{enumerate}
\item 
The Los Andes curriculum reform may be a response to the introduction of college exit exam; as a result the ``treatment" would be endogenous.  (THINK ABOUT THIS) 
\end{enumerate}
These are just a few examples of t. The same problems above can be applied to each of the author's robustness checks as well.   For example, 

\begin{enumerate}

\item \textit{Placebo date}: Crucially, college exit exams were not mandatory for graduation until 2009, and so the patterns observed for cohorts enrolling in 2003-2009 may not hold for later cohorts.  Furthermore, the Although test-taking adoption increased rapidly over this period -- they jumped from 10 percent in 2004, to 55 percent in 2005 -- there may be additional signaling 

\item Los Andes students already enrolled at th etime of the reform but studied under the new curricula -- still affected by the college exit exam
\item Placebo check with law graduates -- had no curriculum reform but also didn't have exit exams at the date of the reform for economics/business students
\item Competing cohorts -- the reform generated two cohorts graduating at the same time, which might ahve distorted wages by creating more competition.  They all would have been subject to the same information assymmetry about the optimal strategy for the exit exam, and correpsonding reduced reputaiton effects.
\item Comparison to next three highest rank schools -- Without controlling for proportion of students taking exam it is not possible to say. 

\end{enumerate}


\section*{Letter to the editor}
Even if it is true, the exit exams introduces a new mechanism by which graduating students' wages may have declined; and this mechanism is related more to signaling that human capital accumulation. 

While she controls for X Y AND Z,  she does not control for the proportion of students taking the exit exams at each university is troubling.   First, the act of taking the exit exam in itself may serve as a signal to employers, as well as the student's relative performance on the exam.  If students at Los Andes were slow to adopt the exam or perform worse, that is a problem

If we do not control for students who choose to take the exam, and also for their performance, it means that the results may be driven by omitted variable bias; the direction of which is unknown.  



Furthermore, there is the issue that the curriculum change at Los Andes was a direct response to the addition of college exit exams.  
At the end of the day, this is a convincing result. I want to believe it.  But without this issue controlled for we have a problem!! May be a bigger bias or not. 

Furthermore, "The results show that in programs with exams, the ability of incoming students
became more correlated with exit exam reputation, and less correlated with Icfes
reputation. In other words, school programs whose exit exam performance exceeded
their average Icfes performance saw increases in the ability of their incoming classes.
This suggests students selected different programs and/or colleges as new information
on their quality became available."


 
 Conclude that they cannot rule out the hypotehsis that college reputation is merely serving as a signal for unobservable characteristics that themselves are related human capital accumulation.  Further, if sortin goccur son the ability dimension, increasing conditional return to reputation could arise because admission scores are imperfect measures of abiity. 
 
IMPERFECT MEASURES OF ABILITY. Regression to the mean
 
Providing more information about student skill reduces the importance of reputation.  Even after controlling for admission scores, a graduate's starting earnings and earnings growth are positively correlated with college reputation. 

The paper tries to do this by analyzing a curriculum reform at Universidad de Los Andes, the top ranked university in Colombia, which reduced the amount of coursework required for graduation by 20 and 14 percent in its economics and business programs, respectively.   Arteaga argues that the reform did not change the quality of incoming or graduating students (ergo they should be the same in ability), and so it serves as a valid natural experiment to test these theoreis.

Arteaga accomplishes what she sets out to do. Her research question is focused ,well investigated, and her results are robust to many specifications.  Hence, her results are convincing.

  She also investigates the mechanism by which wages decline, and concludes that they may result from a decline in performance during the recruitment process, which led students to be placed in lower-quality firms.  The probability of being hired dropped by 17 percentage points in a firm that normally recruits many students.



This paper contributes to several literatures: (1) human capital vs signaling theoretical debate, (2) that which estimates the returns to different types of post-secondary education by separating the effect of the institution versus the field of study (which argues degree type more important than institution). Main contribution is that it provides evidence suggesting that the source of heterogeneity in returns to education may be due to the different sets of skills and knowledge acquired in each degree, and not to differences in selectivity.

Says she is first paper to investigate exact mechanism through which wage decline from less human capital.   contributes to long debate about whether signaling or human capital is responsible for the returns to a university degree. 

\end{document}
