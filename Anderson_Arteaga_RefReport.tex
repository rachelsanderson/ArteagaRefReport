\documentclass[a4paper, 11pt]{article}
\usepackage{comment} % enables the use of multi-line comments (\ifx \fi)
\usepackage{lipsum} %This package just generates Lorem Ipsum filler text.
\usepackage{fullpage} % changes the margin
\usepackage{indentfirst}
\usepackage[utf8]{inputenc}
\usepackage{booktabs}
\usepackage{chngpage}
\setlength{\parindent}{2em}
\renewcommand{\baselinestretch}{1.1} %make line spacing bigger
%custom commands
%\usepackage{pgfplots}
\usepackage[margin=1in]{geometry}

\begin{document}
%Header-Make sure you update this information!!!!

\title{Referee Report for the Quarterly Journal of Economics \\
\Large ``The effect of human capital on earnings: Evidence from a reform at Columbia's top university"}
\author{Rachel Anderson}
\maketitle

\section*{Overview}

The paper tests whether the economic returns to college education result from human capital accumulation or the fact that attending college signals higher ability.  The author does this by studying the wages of students graduating immediately before and after a curriculum reform that reduced the number of required courses for the economics and business programs at a top Colombian university, Universidad de Los Andes.  Following the reform, wages fell by approximately 16\% in economics and 13\% percent in business, which the author attributes to less human capital accumulation, manifested as students being placed in lower-quality firms.  



\section*{A threat to identification}

The author's conclusions hinge on two assumptions: that the curriculum reform did not alter the number or quality of students entering and graduating from Universidad de Los Andes, and that students' capacity to signal their ability did not change in the period surrounding the reform.  If these assumptions hold, then a pure signaling model predicts that the reform should have no effect on wages, whereas the human capital model predicts that wages should decline.  

The author explicitly states and proves that the first assumption is valid, however the second is fatally overlooked.  In particular, the introduction of college exit exams is likely to have changed the signaling structure in Colombia's labor market for new graduates in exactly the years surrounding the curriculum reform. 

In 2004, the agency that administers the college admission exam in Colombia introduced field-specific exit exams to be administered at every college offering a related program.  The stated intent of this change was to make available aggregate school-level statistics about student performance, so as to improve accountability in the college market.  Different degree programs started offering exams in different years, however both business and economics were among the first to offer exams in 2004. 

The Universidad de Los Andes curriculum reform took place in 2006, so that both pre- and post-reform groups would have been exposed to the national exit exams to some extent.  College exit exams did not become mandatory for graduation until 2009, however test-taking increased rapidly from 2004-2009 -- from 10 percent of students graduating in 2004, to 55 percent in 2005.  If students took between 4 and 4.5 years to graduate under the new curriculum, the students entering Los Andes at the times of reform listed in Figure 4 would have also been among the first cohorts required to pass the exit exams for graduation. 

MacLeod et al. (2017) suggest that the exit exams in Colombia served as a new signal of student ability, that in turn led the returns to college reputation, as defined by a school's average college admissions score, to decline.  In particular, students graduating from economics and business programs in 2005-2009 received between 5 and 10 percent lower returns to college reputation, as measured by average daily earnings, compared to cohorts who graduated before the introduction of college exit exams (see Table \ref{tab:macleod}).  The returns to ability, as measured by an individual's college admission score, also declined slightly over this period, as a result of the new signal. 

\begin{table}[h!]
 \begin{adjustwidth}{-.75in}{-.75in} 
 \begin{center}
 \caption{Returns to reputation and ability by program and cohort on log average daily earnings (MacLeod et al. 2017)}
 \vspace{8pt}
\begin{tabular}{cccccccccc}
\toprule
             &                    &        &          & \multicolumn{3}{c}{Return to reputation}&\multicolumn{3}{c}{Return to ability}       \\
             \cmidrule(lr){5-7} \cmidrule(lr){8-10}
Program area & Program            & N      & Colleges & 2003-04              & 2005-09 & Diff. & 2003-04           & 2005-09 & Diff. \\
\midrule
Business     & Administraci\'on     & 85,325 & 46       & 0.18                 & 0.14    & -0.05 & 0.04              & 0.03    & -0.01 \\
Business     & Contadur\'ia p\'ublica & 49,714 & 36       & 0.18                 & 0.09    & -0.09 & 0.03              & 0.03    & 0.00  \\
Business     & Econom\'ia           & 25,879 & 21       & 0.21                 & 0.11    & -0.10 & 0.05              & 0.03    & -0.02 \\
\bottomrule
\multicolumn{8}{l}{\small{*Cohort refers to graduating cohort. }}
\end{tabular}
\label{tab:macleod}
\end{center}
\end{adjustwidth}
\end{table}

The fact that college exit exams improved students' ability to signal quality to employers is supported anecdotally by students who report publishing scores on their CVs and online profiles.  However, the exit exams may have also altered student job seeking, professor recommending, or employer recruiting behaviors in ways that potentially threaten the author's conclusions about the importance of human capital accumulation in determining wages post-graduation.  Below are some competing hypotheses:
\\

\noindent \textbf{\textit{Signaling}}:  The author's baseline database contains all students who enrolled in economics or business programs between 2002 and 2007 and graduated after 2004 from Los Andes and other top-10 schools.  Given the timeline of the curriculum reform, it seems likely that the ``control" groups graduated while the exit exams were not mandatory.  Results from MacLeod et al. suggest that returns to reputation were highest for these students.  

For the ``treatment" group, the decline in wages may be entirely due to decreasing returns to reputation, if the students signaled the same or lower ability as previous cohorts, on average.  One way this could happen is if the students who took the exit exam while it was optional had, on average, higher ability, relative to the population of students at Los Andes and other top-10 schools.  After exams became mandatory, the average exit exam score among Los Andes students would have  declined, thereby decreasing the perception of Los Andes students' ability.  

It is also possible that the curriculum reform at Universidad de Los Andes in 2006 made students less prepared for the exit exams, and so they performed worse relative to students at other top-10 schools and previous Los Andes cohorts.  Alternatively, average exit exam performance at other top-10 schools may have improved over the period, due to university policies that added test preparation sessions or motivated individuals to prepare more for the exams.

All of the scenarios presented above are possible explanations for how Los Andes students, after the curriculum reform, would have signaled ``lower ability" to employers through worse exit exam performance, relative to previous cohorts and to students at other top 10 schools.
  \\
  
\noindent \textbf{\textit{Changes in job search:}} The introduction of exit exams may have changed how students decided where to apply for jobs.  After the reform, the author finds that Los Andes students placed more often in lower-quality firms, and interprets this as evidence of lower applicant quality.  However, it may be the case that students, after observing where they score relative to other students on the exit exams, changed the types of jobs to which they applied.

Prior to the introduction of exit exams, students at Los Andes may have benefited in their job search from connections with alumni or former faculty at top firms.  In the presence of a new signal of ability, the importance of these networks may have declined.  Similarly, faculty may have changed how they choose to recommend students for certain jobs, according to student  performance on the exit exams. 
\\
 
\noindent \textbf{\textit{Employer learning:}} It is possible that firms did not weigh exit exam scores in the hiring process until 2009 (after the curriculum reform), due to the fact that not everyone took the exams.  As a result, the decline in wages may reflect employer learning, as employer's started to understand the relationship between test scores and worker performance. 
  \\
  
\noindent \textbf{\textit{Imperfect measure of ability:}} Without reporting the correlation between students' scores on the two exams, we cannot assume that students performing well on the college admissions exam also perform well on the exit exam.  Furthermore \textit{regression to the mean} suggests that if a variable is extreme on its first measurement (i.e. students who receive top scores on the college admissions exam), will tend to be closer to the average on the second measurement.  



\newpage
\section*{Letter to the editor}
Even if it is true, the exit exams introduces a new mechanism by which graduating s
tudents' wages may have declined; and this mechanism is related more to signaling that human capital accumulation. 

While she controls for X Y AND Z,  she does not control for the proportion of students taking the exit exams at each university is troubling.   First, the act of taking the exit exam in itself may serve as a signal to employers, as well as the student's relative performance on the exam.  If students at Los Andes were slow to adopt the exam or perform worse, that is a problem

If we do not control for students who choose to take the exam, and also for their performance, it means that the results may be driven by omitted variable bias; the direction of which is unknown.  



Furthermore, there is the issue that the curriculum change at Los Andes was a direct response to the addition of college exit exams.  
At the end of the day, this is a convincing result. I want to believe it.  But without this issue controlled for we have a problem!! May be a bigger bias or not. 

Furthermore, "The results show that in programs with exams, the ability of incoming students
became more correlated with exit exam reputation, and less correlated with Icfes
reputation. In other words, school programs whose exit exam performance exceeded
their average Icfes performance saw increases in the ability of their incoming classes.
This suggests students selected different programs and/or colleges as new information
on their quality became available."


 
 Conclude that they cannot rule out the hypotehsis that college reputation is merely serving as a signal for unobservable characteristics that themselves are related human capital accumulation.  Further, if sortin goccur son the ability dimension, increasing conditional return to reputation could arise because admission scores are imperfect measures of abiity. 
 
IMPERFECT MEASURES OF ABILITY. Regression to the mean
 
Providing more information about student skill reduces the importance of reputation.  Even after controlling for admission scores, a graduate's starting earnings and earnings growth are positively correlated with college reputation. 

The paper tries to do this by analyzing a curriculum reform at Universidad de Los Andes, the top ranked university in Colombia, which reduced the amount of coursework required for graduation by 20 and 14 percent in its economics and business programs, respectively.   Arteaga argues that the reform did not change the quality of incoming or graduating students (ergo they should be the same in ability), and so it serves as a valid natural experiment to test these theoreis.

Arteaga accomplishes what she sets out to do. Her research question is focused ,well investigated, and her results are robust to many specifications.  Hence, her results are convincing.

  She also investigates the mechanism by which wages decline, and concludes that they may result from a decline in performance during the recruitment process, which led students to be placed in lower-quality firms.  The probability of being hired dropped by 17 percentage points in a firm that normally recruits many students.



This paper contributes to several literatures: (1) human capital vs signaling theoretical debate, (2) that which estimates the returns to different types of post-secondary education by separating the effect of the institution versus the field of study (which argues degree type more important than institution). Main contribution is that it provides evidence suggesting that the source of heterogeneity in returns to education may be due to the different sets of skills and knowledge acquired in each degree, and not to differences in selectivity.

Says she is first paper to investigate exact mechanism through which wage decline from less human capital.   contributes to long debate about whether signaling or human capital is responsible for the returns to a university degree. 

\section*{Contribution}

This is not the first paper to attempt to estimate the effect of human capital accumulation on wages, separate from that of signaling; however, it is one of the first to do so convincingly, and in a college setting.  Previous papers on the topic focusing on primary and secondary school settings have achieved mixed results, and generally support a signaling framework (\cite{hu16}, \cite{bedard01}, \cite{lang86}).  A related strand of literature seeks to measure directly whether there is signaling value to academic degrees (Clark and Martorell; Murnane and Willett).  Although it cannot resolve these inconsistencies in the literature, this paper contributes a convincing piece of evidence in favor of human capital theory, and provides important information about the tools employers use to learn about workers' expected productivity.  These findings also contribute to a third strand of literature that estimates the returns to different types of post-secondary education by separating the effect of the institution versus the field of study (Dale and Krueger); this paper suggests that heterogeniety in returns may be due to different sets of skills in each degree, not due to differences in selectivity.


\end{document}
